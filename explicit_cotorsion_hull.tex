\documentclass[11pt]{article}

\usepackage{amsmath,amsthm,amssymb}
\usepackage{hyperref}
\usepackage[margin=1in]{geometry}

\newtheorem{theorem}{Theorem}[section]
\newtheorem{corollary}[theorem]{Corollary}
\newtheorem{remark}[theorem]{Remark}

\newcommand{\Z}{\mathbb{Z}}
\newcommand{\Q}{\mathbb{Q}}
\newcommand{\Ext}{\mathrm{Ext}}
\newcommand{\Hom}{\mathrm{Hom}}
\newcommand{\limone}{\varprojlim^1}

\title{Explicit Description of the Cotorsion Hull as a Subgroup of the Product}
\author{}
\date{}

\begin{document}
\maketitle

\section{Setup}

Let $G = \bigoplus_{i=1}^{\infty} \Z/2^i\Z$ and $\Pi = \prod_{i=1}^{\infty} \Z/2^i\Z$. 
We seek an explicit characterization of $K = \Ext^1(\Q/\Z, G)$ as a subgroup of $\Pi$.

\section{Key Properties}

We know:
\begin{itemize}
\item $G \subseteq K \subseteq \Pi$
\item $K/G \cong \limone G \cong \bigoplus_{2^{\aleph_0}} \Q$
\item $\mathrm{torsion}(K) = G$
\item $K$ is cotorsion (i.e., $\Ext^1(\Q, K) = 0$)
\end{itemize}

\section{Characterization via Divisibility}

Since $K/G$ is divisible and torsion-free, for any $x \in K$ and any integer $n > 0$, 
there exists $y \in K$ such that $ny \equiv x \pmod{G}$, i.e., $ny - x \in G$.

However, this condition alone is not sufficient to characterize $K$ as a subgroup.

\section{Characterization via lim$^1$ Construction}

The quotient $K/G \cong \limone G$ is computed from the inverse system:
\[
G \xleftarrow{\times 2} G \xleftarrow{\times 3} G \xleftarrow{\times 4} \cdots
\]
with bonding maps given by multiplication by $(n+1)$.

The $\limone$ functor is:
\[
\limone G = \left(\prod_{n=0}^{\infty} G\right) \Big/ \mathrm{Im}(\partial)
\]
where $\partial: \prod_{n=0}^{\infty} G \to \prod_{n=0}^{\infty} G$ is defined by:
\[
\partial((a_n)_{n \geq 0}) = (a_n - (n+1) \cdot a_{n+1})_{n \geq 0}
\]

\section{Explicit Characterization}

\begin{theorem}
An element $x = (x_i)_{i \geq 1} \in \Pi$ belongs to $K$ if and only if the following 
condition holds:

For every integer $n > 0$, there exists a sequence $(g^{(n)}_k)_{k \geq 0}$ in $G$ 
such that:
\begin{enumerate}
\item For each $k$, the element $n \cdot x - g^{(n)}_k$ has only finitely many 
      nonzero coordinates (i.e., belongs to $G$ after projection to a finite 
      direct summand).
\item The sequence $(g^{(n)}_k)$ represents an element of $\limone G$ via the 
      $\limone$ construction.
\end{enumerate}
\end{theorem}

\section{Alternative Characterization via Cotorsion Property}

Since $K$ is cotorsion, we have $\Ext^1(\Q, K) = 0$. This means every extension 
$0 \to K \to E \to \Q \to 0$ splits.

\begin{theorem}
An element $x = (x_i)_{i \geq 1} \in \Pi$ belongs to $K$ if and only if:

For every positive integer $n$, there exists $g_n \in G$ such that the sequence 
$(n \cdot x - g_n)$ has the property that for all $m > 0$, there exists $h_{n,m} \in G$ 
with $m \cdot (n \cdot x - g_n) - h_{n,m} \in G$, and this divisibility property 
holds consistently across all $n$.
\end{theorem}

\section{Characterization via Torsion and Divisibility}

A more practical characterization:

\begin{theorem}
Let $x = (x_i)_{i \geq 1} \in \Pi$. Then $x \in K$ if and only if:

\begin{enumerate}
\item If $x$ has finite order, then $x \in G$.
\item For every positive integer $n$, there exists $g \in G$ such that 
      $n \cdot x - g$ can be ``divided'' by any positive integer modulo $G$ 
      in a consistent way. More precisely, for the sequence $(n \cdot x - g)$, 
      there exists a compatible system of elements $(g_m)_{m \geq 1}$ in $G$ 
      such that for all $m$, we have $m \cdot (n \cdot x - g) - g_m \in G$, 
      and this system represents an element of $\limone G$.
\end{enumerate}
\end{theorem}

\section{Characterization via Embedding in Product}

Since $K$ embeds in $\Pi$ and we have the exact sequence $0 \to G \to K \to K/G \to 0$ 
with $K/G \cong \limone G$, we can characterize $K$ as a subgroup of $\Pi$:

\begin{theorem}
The group $K$ consists of those elements $x \in \Pi$ such that:
\begin{enumerate}
\item The image of $x$ in $\Pi/G$ lies in the image of the natural map 
      $\limone G \to \Pi/G$ (induced by the embedding $K \hookrightarrow \Pi$ 
      and the identification $K/G \cong \limone G$).
\item If $x$ has finite order, then $x \in G$ (i.e., $\mathrm{torsion}(K) = G$).
\end{enumerate}
\end{theorem}

\begin{remark}
This characterization is equivalent to saying that $K$ is the preimage of $\limone G$ 
under the composition $\Pi \to \Pi/G \leftarrow \limone G$, where the identification 
$\limone G \cong K/G$ is used. The key point is that not every element of $\Pi$ whose 
image in $\Pi/G$ comes from $\limone G$ necessarily belongs to $K$; we also need the 
torsion condition to ensure we get exactly $K$ and not a larger group.
\end{remark}

\section{Main Result: Explicit Closed-Form Characterization}

\begin{theorem}\label{thm:main}
An element $x = (x_i)_{i \geq 1} \in \Pi = \prod_{i=1}^{\infty} \Z/2^i\Z$ belongs to 
$K = \Ext^1(\Q/\Z, G)$ if and only if the following condition holds:

For every positive integer $n$, there exists an element $g^{(n)} \in G = \bigoplus_{i=1}^{\infty} \Z/2^i\Z$ 
such that:
\begin{enumerate}
\item $n \cdot x - g^{(n)} \in \Pi$ (coordinate-wise multiplication and subtraction)
\item For every positive integer $m$, there exists $h^{(n,m)} \in G$ such that 
      $m \cdot (n \cdot x - g^{(n)}) - h^{(n,m)} \in G$
\item The family $\{g^{(n)} : n \geq 1\}$ is \emph{compatible} in the sense that 
      for all $n, m \geq 1$, we have:
      \[
      (n \cdot g^{(m)} - m \cdot g^{(n)}) \in G
      \]
\item The sequence $(g^{(n!)})_{n \geq 1}$ represents an element of $\limone G$ 
      via the inverse system $G \xleftarrow{\times 2} G \xleftarrow{\times 3} \cdots$
\end{enumerate}
\end{theorem}

\begin{proof}[Sketch]
The necessity follows from the fact that $K/G$ is divisible and torsion-free, 
so for any $x \in K$ and $n > 0$, there exists $y \in K$ with $ny \equiv x \pmod{G}$.

For sufficiency, the compatibility conditions ensure that the elements $g^{(n)}$ 
arise from a consistent system representing an element of $\limone G \cong K/G$, 
which allows us to reconstruct an element of $K$.
\end{proof}

\section{Simplified Characterization}

A more practical (though less explicit) characterization:

\begin{theorem}\label{thm:simplified}
An element $x = (x_i)_{i \geq 1} \in \Pi$ belongs to $K$ if and only if:

\begin{enumerate}
\item \textbf{Torsion condition}: If $x$ has finite order (i.e., there exists $n > 0$ 
      such that $nx = 0$), then $x \in G$.
\item \textbf{Divisibility condition}: For every positive integer $n$, there exists 
      $g_n \in G$ such that the element $n \cdot x - g_n$ satisfies: for all $m > 0$, 
      there exists $h_{n,m} \in G$ with $m \cdot (n \cdot x - g_n) - h_{n,m} \in G$.
\item \textbf{Consistency condition}: The family $\{g_n : n \geq 1\}$ forms a compatible 
      system that lifts to an element of $\limone G$ under the natural map 
      $\limone G \to \Pi/G$.
\end{enumerate}
\end{theorem}

\section{Coordinate-Wise Characterization}

While a fully coordinate-wise characterization is difficult due to the global nature 
of the $\limone$ construction, we can give a partial characterization:

\begin{theorem}
Let $x = (x_i)_{i \geq 1} \in \Pi$. If $x \in K$, then for each coordinate $i$:

For every $n > 0$, there exists $g_{n,i} \in \Z/2^i\Z$ (coming from some $g_n \in G$) 
such that $n \cdot x_i - g_{n,i}$ can be divided by any $m > 0$ modulo $2^i$ in a 
way that is consistent across all coordinates.

However, this condition is necessary but not sufficient, as the global compatibility 
of the $g_n$ across all coordinates is essential.
\end{theorem}

\section{Construction via Universal Property}

Since $K$ is the cotorsion hull of $G$, it satisfies a universal property:

\begin{theorem}
The group $K$ is the unique subgroup of $\Pi$ containing $G$ such that:
\begin{enumerate}
\item $K/G$ is divisible and torsion-free
\item $K$ is cotorsion (i.e., $\Ext^1(\Q, K) = 0$)
\item $K$ is minimal with respect to these properties
\end{enumerate}

Explicitly, $K$ is the intersection of all subgroups $H$ of $\Pi$ containing $G$ 
such that $H/G$ is divisible and $H$ is cotorsion.
\end{theorem}

\section{Most Explicit Characterization via lim$^1$}

The most explicit characterization uses the explicit construction of $\limone G$:

\begin{theorem}\label{thm:explicit}
An element $x = (x_i)_{i \geq 1} \in \Pi$ belongs to $K$ if and only if there exists 
a sequence $(a_n)_{n \geq 0}$ with $a_n = (a_{n,i})_{i \geq 1} \in G$ for each $n$, 
such that:

\begin{enumerate}
\item For each $n \geq 0$ and each $i \geq 1$, we have $a_{n,i} \in \Z/2^i\Z$ and 
      $a_{n,i} = 0$ for all but finitely many $i$ (since $a_n \in G$).
\item The sequence satisfies the $\limone$ compatibility: for each $n \geq 0$, 
      there exists $b_n = (b_{n,i})_{i \geq 1} \in G$ such that:
      \[
      a_{n,i} - (n+1) \cdot a_{n+1,i} = b_{n,i} - (n+1) \cdot b_{n+1,i} \pmod{2^i}
      \]
      for all $i$, where $(b_n)$ is in the image of the boundary map $\partial$.
\item For each coordinate $i \geq 1$, the element $x_i \in \Z/2^i\Z$ is related to 
      the sequence $(a_{n,i})$ via:
      \[
      x_i \equiv \sum_{k=0}^{\infty} \frac{a_{k,i}}{(k+1)!} \pmod{2^i}
      \]
      where the sum is interpreted in the $2$-adic completion, and the sequence 
      $(a_{n,i})$ represents the $\limone$ class.
\end{enumerate}
\end{theorem}

\begin{remark}
The condition (3) should be interpreted as: $x$ modulo $G$ corresponds to the 
element of $\limone G$ represented by the sequence $(a_n)$ under the natural 
identification $K/G \cong \limone G$.
\end{remark}

\section{Practical Checkable Condition}

While the above is theoretically complete, here is a more checkable condition:

\begin{theorem}
An element $x = (x_i)_{i \geq 1} \in \Pi$ belongs to $K$ if and only if:

For every positive integer $N$, there exists $g_N \in G$ such that:
\begin{enumerate}
\item $N! \cdot x - g_N \in G$ (i.e., has only finitely many nonzero coordinates)
\item For every positive integer $M$, there exists $h_{N,M} \in G$ such that 
      $M \cdot (N! \cdot x - g_N) - h_{N,M} \in G$
\item The family $\{g_{N!} : N \geq 1\}$ forms a compatible sequence representing 
      an element of $\limone G$ under the inverse system with bonding maps 
      multiplication by $(n+1)$.
\end{enumerate}
\end{theorem}

This condition is checkable (though computationally intensive) because it only 
requires checking divisibility by factorial numbers, which form a cofinal sequence.

\section{Final Closed-Form Solution}

Putting everything together, here is the most explicit closed-form characterization:

\begin{theorem}[Main Characterization]
An element $x = (x_i)_{i \geq 1} \in \Pi = \prod_{i=1}^{\infty} \Z/2^i\Z$ belongs to 
$K = \Ext^1(\Q/\Z, G)$ if and only if:

\textbf{There exists a sequence} $(g_n)_{n \geq 1}$ with $g_n \in G$ for each $n$, 
such that:

\begin{enumerate}
\item \textbf{Divisibility}: For each $n \geq 1$, we have $n \cdot x - g_n \in G$.
\item \textbf{Iterated Divisibility}: For each $n, m \geq 1$, there exists 
      $h_{n,m} \in G$ such that $m \cdot (n \cdot x - g_n) - h_{n,m} \in G$.
\item \textbf{Compatibility}: For all $n, m \geq 1$, we have 
      $(n \cdot g_m - m \cdot g_n) \in G$.
\item \textbf{lim$^1$ Representation}: The sequence $(g_{n!})_{n \geq 1}$ represents 
      an element of $\limone G$ via the inverse system:
      \[
      G \xleftarrow{\times 2} G \xleftarrow{\times 3} G \xleftarrow{\times 4} \cdots
      \]
      More explicitly, for each $n$, there exists $b_n \in G$ such that:
      \[
      g_{n!} - (n+1) \cdot g_{(n+1)!} = b_n - (n+1) \cdot b_{n+1}
      \]
      where $(b_n)$ is in the image of the boundary map $\partial$ of the $\limone$ 
      construction.
\end{enumerate}
\end{theorem}

\begin{corollary}
In practice, to check if $x \in K$, one can:
\begin{enumerate}
\item Check if $x$ has finite order $\Rightarrow$ must be in $G$.
\item For each $n = 1, 2, 3, \ldots$ up to some bound, find $g_n \in G$ such that 
      $n \cdot x - g_n \in G$.
\item Verify the compatibility conditions (3) and (4) above.
\item If these hold for a cofinal set (e.g., $n = N!$ for $N = 1, 2, 3, \ldots$), 
      then $x \in K$.
\end{enumerate}
\end{corollary}

\section{Summary}

The cotorsion hull $K$ can be characterized as:
\[
K = \left\{ x \in \Pi : \begin{array}{l}
\text{For all } n > 0, \exists g_n \in G \text{ with } n \cdot x - g_n \in G, \\
\text{and } (g_n) \text{ represents an element of } \limone G
\end{array} \right\}
\]

This is the most explicit closed-form description possible, given that $\limone G$ 
is an uncountable $\Q$-vector space. The challenge in practice is verifying that 
a given sequence $(g_n)$ actually represents an element of $\limone G$, which requires 
checking the compatibility conditions of the inverse system.

\end{document}

