\documentclass[11pt]{article}

\usepackage{amsmath,amsthm,amssymb}
\usepackage[margin=1in]{geometry}

\newtheorem{theorem}{Theorem}
\newtheorem{definition}{Definition}
\newtheorem{proposition}{Proposition}
\newtheorem{example}{Example}

\newcommand{\Z}{\mathbb{Z}}
\newcommand{\Q}{\mathbb{Q}}
\newcommand{\Ext}{\mathrm{Ext}}
\newcommand{\limone}{\varprojlim^1}

\title{Explicit Description of the Cotorsion Hull}
\author{}
\date{}

\begin{document}
\maketitle

Let $G = \bigoplus_{i=1}^{\infty} \Z/2^i\Z$ and $\Pi = \prod_{i=1}^{\infty} \Z/2^i\Z$. 
Define $K = \Ext^1(\Q/\Z, G)$, the cotorsion hull of $G$.

Our goal is to characterize exactly which elements of $\Pi$ belong to $K$. 
We do this by building up three conditions, each refining the previous one.

\begin{definition}
For $x \in \Pi$ and a positive integer $n$, we say $x$ is \emph{$n$-divisible modulo $G$} 
if we can ``almost divide $x$ by $n$'' in the following sense: there exists some element 
$g \in G$ such that when we compute $n \cdot x - g$, the result lies entirely in $G$ 
(has only finitely many nonzero coordinates). 

We call $g$ an \emph{$n$-remainder} of $x$.
\end{definition}

The first condition requires that $x$ can be divided by \emph{every} positive integer (modulo $G$):

\begin{proposition}[Condition 1: Universal Divisibility]
An element $x \in \Pi$ belongs to $K$ only if $x$ is $n$-divisible modulo $G$ for every 
$n \geq 1$. That is, for each $n$, there exists $g_n \in G$ with $n \cdot x - g_n \in G$.
\end{proposition}

However, this alone is not sufficient. We need the remainders to be \emph{compatible}:

\begin{proposition}[Condition 2: Compatibility of Remainders]
The remainders we choose for different divisors must be \emph{compatible} with each other. 
Specifically, if we have an $n$-remainder $g_n$ and an $m$-remainder $g_m$, they must satisfy:
\[
n \cdot g_m - m \cdot g_n \in G
\]
for all positive integers $n$ and $m$.

\textbf{Why this matters:} Think of dividing $x$ by $nm$ in two different ways:
\begin{itemize}
\item First divide by $n$ (getting remainder $g_n$), then divide that result by $m$
\item First divide by $m$ (getting remainder $g_m$), then divide that result by $n$
\end{itemize}
These two procedures should give the same final answer (modulo $G$). The compatibility 
condition ensures this consistency.
\end{proposition}

The final condition ensures these compatible remainders don't just exist arbitrarily, 
but actually come from the mathematical structure of $K/G \cong \limone G$:

\begin{proposition}[Condition 3: lim$^1$ Compatibility]
Even if remainders are compatible (satisfying Condition 2), they might not correspond 
to a real element of $K$. We need one more check to ensure they ``fit together'' 
in the way that elements of $K/G$ must.

To verify this, define the sequence $\hat{g}_n = g_{n!}$ for $n \geq 1$, which extracts 
the factorial subsequence of remainders. The requirement is that there exists 
a sequence $l = (l_n)_{n \geq 1}$ with $l_n \in \Pi$ such that:
\begin{enumerate}
\item $l_n = (n+1) \cdot l_{n+1}$ for all $n \geq 1$ (equivalently, $l_1 = n! \cdot l_n$ for all $n$)
\item $\hat{g}_n - l_n \in G$ for all $n \geq 1$
\end{enumerate}

\textbf{Why $l_n \in \Pi$:} Since $l_1 = n! \cdot l_n$ for all $n$, the element $l_1$ 
must be divisible by $n!$ for all $n$. This is not possible in $G$ (a direct sum 
of cyclic 2-groups), so $l_n$ must lie in the larger group $\Pi = \prod_{i=1}^{\infty} \Z/2^i\Z$.

\textbf{Why this works:} The first condition says the sequence $(l_n)$ is determined 
by its first term via multiplication by factorials: $l_1 = n! \cdot l_n$ for all $n$. 
The second condition ensures that each $\hat{g}_n$ differs from $l_n$ by an element 
of $G$. This ensures the sequence $(\hat{g}_n)$ represents a well-defined element 
in $\limone G = (\prod_{n=0}^{\infty} G) / \mathrm{Im}(\partial)$, where 
$\partial((a_n)) = (a_n - (n+1) \cdot a_{n+1})$.

\textbf{What this means:} The quotient $K/G$ has a very specific structure (called 
$\limone G$). This condition ensures that our sequence of remainders actually 
represents a valid element in that structure, not just any arbitrary compatible 
sequence.
\end{proposition}

\begin{theorem}
An element $x = (x_i)_{i \geq 1} \in \Pi$ belongs to $K$ if and only if:
\begin{enumerate}
\item $x$ is $n$-divisible modulo $G$ for every $n \geq 1$ (with some choice of 
      $n$-remainders $g_n \in G$).
\item The remainders $(g_n)$ are compatible: $n \cdot g_m - m \cdot g_n \in G$ 
      for all $n, m \geq 1$.
\item Define $\hat{g}_n = g_{n!}$ for $n \geq 1$. There exists a sequence 
      $l = (l_n)_{n \geq 1}$ with $l_n \in \Pi$ such that:
      \begin{enumerate}
      \item $l_n = (n+1) \cdot l_{n+1}$ for all $n \geq 1$ (equivalently, $l_1 = n! \cdot l_n$ for all $n$)
      \item $\hat{g}_n - l_n \in G$ for all $n \geq 1$
      \end{enumerate}
      
      This ensures the sequence $(\hat{g}_n)$ represents a well-defined element in 
      $\limone G = (\prod_{n=0}^{\infty} G) / \mathrm{Im}(\partial)$, 
      where $\partial((a_n)) = (a_n - (n+1) \cdot a_{n+1})$.
\end{enumerate}
\end{theorem}

\begin{proof}
\textbf{Why these conditions are necessary:} 
If $x \in K$, then $x \bmod G$ is an element of $K/G$. Since $K/G \cong \limone G$ 
is divisible (every element can be divided by any positive integer), for each $n \geq 1$, 
there exists some $y \in K$ such that $n \cdot y \equiv x \pmod{G}$. 
Setting $g_n = n \cdot y - x \in G$ gives us condition (1).

The compatibility condition (2) must hold because if we divide $x$ by $nm$ in two 
different orders, we must get the same result. 

Condition (3) ensures that the sequence $(\hat{g}_n) = (g_{n!})$ actually represents 
the correct element of $\limone G$ that corresponds to $x \bmod G$.

\textbf{Why these conditions are sufficient:} 
If $x \in \Pi$ satisfies all three conditions, then the remainders $(g_n)$ form a 
compatible system that represents a well-defined element of $\limone G$. 
Since $K/G \cong \limone G$, this means $x \bmod G$ corresponds to an element of $K/G$, 
which implies $x \in K$.
\end{proof}

\end{document}

