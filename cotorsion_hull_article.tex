\documentclass[11pt]{article}

\usepackage{amsmath,amsthm,amssymb}
\usepackage{hyperref}
\usepackage[margin=1in]{geometry}

% Theorem environments
\newtheorem{theorem}{Theorem}[section]
\newtheorem{lemma}[theorem]{Lemma}
\newtheorem{proposition}[theorem]{Proposition}
\newtheorem{corollary}[theorem]{Corollary}
\theoremstyle{definition}
\newtheorem{definition}[theorem]{Definition}
\newtheorem{example}[theorem]{Example}
\newtheorem{remark}[theorem]{Remark}
\newtheorem{question}[theorem]{Open Question}

% Notation
\newcommand{\Z}{\mathbb{Z}}
\newcommand{\Q}{\mathbb{Q}}
\newcommand{\Ext}{\mathrm{Ext}}
\newcommand{\Hom}{\mathrm{Hom}}
\newcommand{\yes}{\checkmark}
\newcommand{\no}{$\times$}

\title{On the Cotorsion Hull of an Unbounded Direct Sum of Cyclic $p$-Groups}
\author{}
\date{}

\begin{document}
\maketitle

\begin{abstract}
We investigate the structure of $\Ext^1(\Q/\Z, G)$ where $G = \bigoplus_{i=1}^{\infty} \Z/2^i\Z$ 
is the direct sum of cyclic 2-groups of unbounded order. This group, known as the cotorsion hull 
of $G$, exhibits subtle properties: it is cotorsion with torsion subgroup exactly $G$, yet it 
is not pure-injective. We compute the quotient $K/G$ as a $\Q$-vector space of uncountable 
dimension and characterize the group via Ulm invariants. We also derive explicit 2-cocycle 
representatives for elements in the image of the connecting homomorphism.
\end{abstract}

\section{Introduction}

Let $G = \bigoplus_{i=1}^{\infty} \Z/2^i\Z$ be the direct sum of cyclic groups of orders 
$2, 4, 8, 16, \ldots$, and let $H = \Q/\Z$ denote the Pr\"ufer group (rationals modulo integers). 
We seek to compute $\Ext^1(H, G)$.

This problem lies at the intersection of several classical themes in abelian group theory: 
cotorsion groups, algebraic compactness, and the derived functors $\lim^1$. The group $G$ 
has \emph{unbounded torsion}, meaning there is no integer $n$ with $nG = 0$, which creates 
delicate issues when considering its various completions.

\subsection{Main Results}

\begin{theorem}\label{thm:main}
The group $K = \Ext^1(\Q/\Z, G)$ is the unique adjusted cotorsion group with torsion subgroup 
isomorphic to $G$. It fits into a non-split exact sequence
\[
0 \to G \to K \to D \to 0
\]
where $D = \Ext^1(\Q, G)$ is a torsion-free divisible group of uncountable dimension.
\end{theorem}

\begin{theorem}\label{thm:quotient}
The quotient $K/G \cong \Ext^1(\Q, G) = \lim^1 G$ is isomorphic as an abstract group to 
$\bigoplus_{2^{\aleph_0}} \Q$, a direct sum of continuum-many copies of the rationals.
\end{theorem}

\section{Preliminaries}

\subsection{Cotorsion Groups}

\begin{definition}[\cite{Fuchs1970}]
An abelian group $C$ is \emph{cotorsion} if $\Ext^1(F, C) = 0$ for all torsion-free groups $F$. 
Equivalently, $\Ext^1(\Q, C) = 0$.
\end{definition}

\begin{definition}
A cotorsion group is \emph{adjusted} if it has no nonzero torsion-free direct summands.
\end{definition}

The following theorem establishes a fundamental correspondence.

\begin{theorem}[\cite{Harrison1959}, \cite{Fuchs1973}]\label{thm:correspondence}
There is a one-to-one correspondence between adjusted cotorsion groups and reduced torsion groups, 
given by $C \mapsto tC$ (the torsion subgroup of $C$).
\end{theorem}

\subsection{The Cotorsion Hull}

\begin{definition}
For a reduced abelian group $A$, the \emph{cotorsion hull} of $A$ is defined as $\Ext^1(\Q/\Z, A)$.
\end{definition}

\begin{proposition}[\cite{Fuchs1973}]\label{prop:hull-properties}
For a reduced torsion group $T$:
\begin{enumerate}
\item $\Ext^1(\Q/\Z, T)$ is cotorsion.
\item The torsion subgroup of $\Ext^1(\Q/\Z, T)$ equals $T$.
\item The quotient $\Ext^1(\Q/\Z, T)/T$ is torsion-free and divisible.
\end{enumerate}
\end{proposition}

\subsection{The lim$^1$ Functor}

For an inverse system of abelian groups $\cdots \to A_{n+1} \xrightarrow{f_n} A_n \to \cdots$, 
the derived functor $\lim^1$ is computed as \cite{Weibel1994}:
\[
\lim^1 A_n = \left(\prod_{n=0}^{\infty} A_n\right) \Big/ \mathrm{Im}(\partial)
\]
where $\partial((a_n)) = (a_n - f_n(a_{n+1}))_{n \geq 0}$.

\section{The Fundamental Exact Sequence}

\subsection{Derivation}

From the short exact sequence $0 \to \Z \to \Q \to \Q/\Z \to 0$, applying $\Hom(-, G)$ yields:
\[
\cdots \to \Hom(\Q, G) \to \Hom(\Z, G) \xrightarrow{\delta} \Ext^1(\Q/\Z, G) \to \Ext^1(\Q, G) \to \Ext^1(\Z, G) \to \cdots
\]

\begin{lemma}\label{lem:vanishing}
For $G = \bigoplus_i \Z/2^i\Z$:
\begin{enumerate}
\item $\Hom(\Q, G) = 0$ \quad (since $\Q$ is divisible and $G$ is reduced torsion)
\item $\Hom(\Z, G) \cong G$
\item $\Ext^1(\Z, G) = 0$ \quad (since $\Z$ is projective)
\end{enumerate}
\end{lemma}

\begin{corollary}
There is a short exact sequence
\[
0 \to G \xrightarrow{\delta} \Ext^1(\Q/\Z, G) \to \Ext^1(\Q, G) \to 0
\]
where $\delta$ is the connecting homomorphism.
\end{corollary}

\subsection{Non-Splitting}

\begin{proposition}\label{prop:nonsplit}
The sequence $0 \to G \to K \to K/G \to 0$ does not split.
\end{proposition}

\begin{proof}
Suppose the sequence splits, so $K \cong G \oplus (K/G)$. Since direct summands of cotorsion 
groups are cotorsion, $G$ would be cotorsion. However, a torsion group is cotorsion if and 
only if it is the direct sum of a divisible group and a bounded group \cite[Theorem 54.1]{Fuchs1970}. 
Since $G = \bigoplus_i \Z/2^i\Z$ is reduced and has unbounded exponent, it is not cotorsion. Contradiction.
\end{proof}

\section{Computing the Quotient $K/G$}

\subsection{Identification with lim$^1$}

\begin{theorem}\label{thm:lim1}
$\Ext^1(\Q, G) \cong \lim^1 G$ where the inverse system is
\[
G \xleftarrow{\times 2} G \xleftarrow{\times 3} G \xleftarrow{\times 4} \cdots
\]
with the $n$-th map being multiplication by $(n+1)$.
\end{theorem}

\begin{proof}
Write $\Q = \varinjlim_{n} \frac{1}{n!}\Z$. Then by \cite[Proposition 3.5.8]{Weibel1994}:
\[
\Ext^1(\Q, G) = \Ext^1(\varinjlim \tfrac{1}{n!}\Z, G) \cong \lim^1 \Hom(\tfrac{1}{n!}\Z, G) = \lim^1 G
\]
where the bonding maps are induced by the inclusions $\frac{1}{n!}\Z \hookrightarrow \frac{1}{(n+1)!}\Z$.
\end{proof}

\subsection{Structure of the Quotient}

\begin{proposition}\label{prop:divisible}
$K/G = \Ext^1(\Q, G)$ is torsion-free and divisible.
\end{proposition}

\begin{proof}
\emph{Divisibility}: For any integer $m$, consider the endomorphism $\times m: \Q \to \Q$, which 
is an isomorphism. This induces an isomorphism $\Ext^1(\Q, G) \xrightarrow{\cong} \Ext^1(\Q, G)$, 
which is multiplication by $m$. Hence $m \cdot \Ext^1(\Q, G) = \Ext^1(\Q, G)$.

\emph{Torsion-free}: If $x \in K/G$ satisfies $mx = 0$ for some $m > 0$, lift $x$ to $\tilde{x} \in K$. 
Then $m\tilde{x} \in G$, so $\tilde{x}$ has finite order in $K$. Since $\mathrm{torsion}(K) = G$, 
we have $\tilde{x} \in G$, so $x = 0$.
\end{proof}

\begin{theorem}\label{thm:cardinality}
The group $K$ is uncountable, with cardinality $|K| = 2^{\aleph_0}$.
\end{theorem}

\begin{proof}
Since $K$ fits into the exact sequence $0 \to G \to K \to \Ext^1(\Q, G) \to 0$ and $G$ is countable, 
it suffices to show that $\Ext^1(\Q, G) \cong \lim^1 G$ has cardinality $2^{\aleph_0}$.

The inverse system defining $\lim^1 G$ is $G \xleftarrow{\times (n+1)} G$. Since $G = \bigoplus_i \Z/2^i\Z$ 
is unbounded, the image of the bonding map $G \xrightarrow{\times m} G$ is strictly smaller than $G$ 
for large $m$. Specifically, the system does not satisfy the Mittag-Leffler condition. 

By a theorem of Jensen (1972), for an inverse system of countable abelian groups, the derived limit $\lim^1$ 
is either trivial or has cardinality $2^{\aleph_0}$. Since the maps are not surjective (leaving ``holes'' 
that accumulate), $\lim^1 G \neq 0$. Being a vector space over $\Q$, it must therefore have uncountable 
dimension, so $|\Ext^1(\Q, G)| = 2^{\aleph_0}$.
\end{proof}

\section{Comparison with Pure-Injective Envelope}

\subsection{Torsion-Complete Groups}

\begin{definition}[\cite{Fuchs1973}]
For a direct sum of cyclic $p$-groups $B = \bigoplus_i B_i$, the \emph{torsion-complete} group 
is the torsion subgroup of the $p$-adic completion:
\[
\bar{B} = \{(a_i) \in \prod_i B_i : \exists k, \; p^k a_i = 0 \text{ for all } i\}
\]
\end{definition}

\begin{theorem}[\cite{Fuchs1973}]\label{thm:pure-inj}
A reduced $p$-group is algebraically compact (= pure-injective) if and only if it is torsion-complete.
\end{theorem}

\subsection{Why $K \neq \bar{G}$}

\begin{proposition}
The cotorsion hull $K = \Ext^1(\Q/\Z, G)$ is not equal to the torsion-complete group $\bar{G}$.
\end{proposition}

\begin{proof}
The torsion-complete group $\bar{G}$ is:
\[
\bar{G} = \{(a_i) \in \prod_i \Z/2^i\Z : \exists k, \; 2^k a_i = 0 \text{ for all } i\}
\]
This is a torsion group strictly larger than $G = \bigoplus_i \Z/2^i\Z$. For example, the 
element $(1, 1, 1, \ldots)$ where each component is $1 \in \Z/2^i\Z$ lies in $\bar{G}$ 
(with $k = 1$ since $2 \cdot 1 = 0$ in $\Z/2$), but not in $G$.

Since $\mathrm{torsion}(\bar{G}) = \bar{G} \supsetneq G$ but $\mathrm{torsion}(K) = G$, we have $K \neq \bar{G}$.
\end{proof}

\begin{remark}
The cotorsion hull $K$ and the torsion-complete group $\bar{G}$ are \emph{incomparable} 
in the containment order. Specifically:
\begin{itemize}
\item $K \not\subset \bar{G}$: The quotient $K/G$ is torsion-free, so $K$ contains elements 
of infinite order. But $\bar{G}$ is entirely torsion, so these elements cannot lie in $\bar{G}$.
\item $\bar{G} \not\subset K$: The group $\bar{G}$ contains torsion elements outside $G$ (e.g., 
$(1,1,1,\ldots)$), but $\mathrm{torsion}(K) = G$, so these cannot lie in $K$.
\end{itemize}
Their intersection is exactly $G$:
\[
K \cap \bar{G} = G
\]
Both $K$ and $\bar{G}$ embed into the full product $\prod_i \Z/2^i\Z$, but they capture 
different ``completions'' of $G$. 

\textbf{Note:} Since $K$ is the minimal cotorsion group containing $G$, if the torsion-complete group $\bar{G}$ were cotorsion, it would necessarily contain $K$. The fact that $K \not\subset \bar{G}$ thus provides a proof that $\bar{G}$ is not cotorsion.
\end{remark}

\begin{proposition}
Let $\Pi = \prod_{i=1}^\infty \Z/2^i\Z$ and let $R \cong \prod_{\aleph_0} \mathbb{Z}_2$ denote the reduced torsion-free algebraically compact part. The structure of the global quotients is:
\begin{enumerate}
    \item $K/G \cong \bigoplus_{2^{\aleph_0}} \Q$
    \item $\bar{G}/G \cong \bigoplus_{2^{\aleph_0}} \Z(2^\infty)$
    \item $\Pi/K \cong R \oplus \bigoplus_{2^{\aleph_0}} \Z(2^\infty)$
    \item $\Pi/\bar{G} \cong \bigoplus_{2^{\aleph_0}} \Q \oplus R$ \quad (note: $\bigoplus_{2^{\aleph_0}} \Q \cong \Q^{\oplus 2^{\aleph_0}}$)
    \item $\Pi/(K+\bar{G}) \cong R$
    \item $\Pi/G \cong \bigoplus_{2^{\aleph_0}} \Q \oplus R \oplus \bigoplus_{2^{\aleph_0}} \Z(2^\infty)$
\end{enumerate}
\end{proposition}

\section{Explicit Cocycles}

\subsection{Extensions as 2-Cocycles}

An element of $\Ext^1(A, B)$ corresponds to an extension $0 \to B \to E \to A \to 0$. Given a 
set-theoretic section $s: A \to E$, the \emph{2-cocycle} $c: A \times A \to B$ is
\[
c(a_1, a_2) = s(a_1) + s(a_2) - s(a_1 + a_2)
\]

\subsection{The Connecting Homomorphism}

The image of the connecting homomorphism $\delta: G \to K$ consists of explicit cocycles.

\begin{proposition}
For $g \in G$, the cocycle $c_g: \Q/\Z \times \Q/\Z \to G$ representing $\delta(g)$ is:
\[
c_g([r_1], [r_2]) = \epsilon(r_1, r_2) \cdot g
\]
where $r_1, r_2 \in [0, 1)$ are representatives and
\[
\epsilon(r_1, r_2) = \begin{cases} 1 & \text{if } r_1 + r_2 \geq 1 \\ 0 & \text{otherwise} \end{cases}
\]
is the ``carry'' function.
\end{proposition}

\section{Ulm Invariant Characterization}

\subsection{Ulm's Theorem}

For a reduced abelian $p$-group $T$, the $\alpha$-th Ulm invariant is \cite{Fuchs1970}:
\[
u_\alpha(T) = \dim_{\mathbb{F}_p} \frac{(p^\alpha T)[p]}{(p^{\alpha+1} T)[p]}
\]

\begin{theorem}[Ulm's Theorem, \cite{Ulm1933}]
Two countable reduced abelian $p$-groups are isomorphic if and only if they have the same Ulm invariants.
\end{theorem}

\subsection{Application to $G$}

\begin{proposition}
For $G = \bigoplus_{i=1}^{\infty} \Z/2^i\Z$:
\[
u_0(G) = \aleph_0, \quad u_\alpha(G) = 0 \text{ for } \alpha \geq 1
\]
\end{proposition}

\begin{corollary}
By Theorem \ref{thm:correspondence}, the cotorsion hull $K = \Ext^1(\Q/\Z, G)$ is the unique 
adjusted cotorsion group with Ulm invariants $u_0 = \aleph_0$ and $u_\alpha = 0$ for $\alpha \geq 1$.
\end{corollary}

\section{Complete Lattice Analysis}

All groups embed into $\Pi = \prod_{i=1}^{\infty} \Z/2^i\Z$. The containment lattice is:
\begin{center}
\begin{tabular}{c}
$\Pi$ \\[2pt]
$|$ \\[2pt]
$K + \bar{G}$ \\[2pt]
$\diagup \quad \diagdown$ \\[2pt]
$K \qquad\qquad \bar{G}$ \\[2pt]
$\diagdown \quad \diagup$ \\[2pt]
$G$ \\[2pt]
$|$ \\[2pt]
$0$
\end{tabular}
\end{center}

\subsection{Structural Properties}

\begin{center}
\begin{tabular}{|l|c|c|c|c|c|c|}
\hline
\textbf{Group} & \textbf{Torsion} & \textbf{TF} & \textbf{Mixed} & \textbf{Bounded} & \textbf{Countable} & \textbf{Card.} \\
\hline
$G$ & \yes & \no & \no & \no & \yes & $\aleph_0$ \\
$K$ & \no & \no & \yes & \no & \no & $2^{\aleph_0}$ \\
$\bar{G}$ & \yes & \no & \no & per elt & \no & $2^{\aleph_0}$ \\
$K+\bar{G}$ & \no & \no & \yes & \no & \no & $2^{\aleph_0}$ \\
$\Pi$ & \no & \no & \yes & \no & \no & $2^{\aleph_0}$ \\
\hline
\end{tabular}
\end{center}

\subsection{Homological Properties}

\begin{center}
\begin{tabular}{|l|c|c|c|c|c|c|}
\hline
\textbf{Group} & \textbf{Div} & \textbf{Red} & \textbf{Cotor} & \textbf{Pure-Inj} & \textbf{Inj} & \textbf{Proj} \\
\hline
$G$ & \no & \yes & \no & \no & \no & \no \\
$K$ & \no & \yes & \yes & \no & \no & \no \\
$\bar{G}$ & \no & \yes & \no & \yes & \no & \no \\
$K+\bar{G}$ & \no & \yes & \no & \no & \no & \no \\
$\Pi$ & \no & \yes & \yes & \yes & \no & \no \\
\hline
\end{tabular}
\end{center}

\subsection{Completeness Properties}

\begin{center}
\begin{tabular}{|l|c|c|c|c|}
\hline
\textbf{Group} & \textbf{$\Z$-adic Compl} & \textbf{$2$-adic Compl} & \textbf{Tor-Compl} & \textbf{Alg Compact} \\
\hline
$G$ & \no & \no & \no & \no \\
$K$ & \no & \no & \no & \no \\
$\bar{G}$ & \no & \no & \yes & \yes \\
$K+\bar{G}$ & \no & \no & \no & \no \\
$\Pi$ & \yes & \yes & N/A & \yes \\
\hline
\end{tabular}
\end{center}

\subsection{Properties of Quotients by $G$}

\begin{center}
\begin{tabular}{|l|c|c|c|c|c|c|}
\hline
\textbf{Quotient} & \textbf{Torsion} & \textbf{TF} & \textbf{Div} & \textbf{Red} & \textbf{Cotor} & \textbf{Inj} \\
\hline
$K/G$ & \no & \yes & \yes & \no & \yes & \yes \\
$\bar{G}/G$ & \yes & \no & \yes & \no & \yes & \no \\
$(K+\bar{G})/G$ & mixed & \no & \yes & \no & \yes & \no \\
$\Pi/G$ & mixed & \no & \no & \no & ? & \no \\
\hline
\end{tabular}
\end{center}

\subsection{Lattice Structure}

The relationships between the groups form a non-distributive lattice structure. The non-trivial structural facts are:

\begin{itemize}
    \item \textbf{Incomparability}: $K \not\subset \bar{G}$ and $\bar{G} \not\subset K$.
    \item \textbf{Intersection}: $K \cap \bar{G} = G$.
    \item \textbf{Strict Containments}:
    \begin{itemize}
        \item $G \subsetneq K \subsetneq K+\bar{G} \subsetneq \Pi$
        \item $G \subsetneq \bar{G} \subsetneq K+\bar{G} \subsetneq \Pi$
    \end{itemize}
    \item \textbf{The Join}: The sum $K+\bar{G}$ is the smallest group containing both completions, but it is still strictly smaller than the full product $\Pi$.
\end{itemize}

\section{Open Questions}


\begin{question}
Can one give an explicit set-theoretic description of $K$ as a subgroup of $\prod_i \Z/2^i\Z$? 
The naive characterization ``$x$ has finite order implies $x \in \bigoplus_i \Z/2^i\Z$'' fails 
to define a subgroup.
\end{question}

\begin{question}
What is the structure of the $\lim^1$ cocycles explicitly? Can one describe elements of 
$K \setminus \delta(G)$ via concrete 2-cocycles $\Q/\Z \times \Q/\Z \to G$?
\end{question}

\begin{question}
For which reduced torsion groups $T$ does the cotorsion hull $\Ext^1(\Q/\Z, T)$ coincide 
with the pure-injective envelope? The case $T = G$ shows they can differ.
\end{question}

\begin{question}
What is the natural topology on $K$ induced by its various completions, and how does it 
relate to the $2$-adic topology on $G$?
\end{question}

\begin{thebibliography}{99}

\bibitem{CartanEilenberg1956}
H.~Cartan and S.~Eilenberg,
\emph{Homological Algebra},
Princeton University Press, Princeton, NJ, 1956.

\bibitem{Fuchs1970}
L.~Fuchs,
\emph{Infinite Abelian Groups, Vol.~I},
Academic Press, New York, 1970.

\bibitem{Fuchs1973}
L.~Fuchs,
\emph{Infinite Abelian Groups, Vol.~II},
Academic Press, New York, 1973.

\bibitem{Harrison1959}
D.~K.~Harrison,
Infinite abelian groups and homological methods,
\emph{Ann. of Math.} (2) \textbf{69} (1959), 366--391.

\bibitem{Ulm1933}
H.~Ulm,
Zur Theorie der abz\"ahlbar-unendlichen Abelschen Gruppen,
\emph{Math. Ann.} \textbf{107} (1933), 774--803.

\bibitem{Weibel1994}
C.~A.~Weibel,
\emph{An Introduction to Homological Algebra},
Cambridge Studies in Advanced Mathematics, vol.~38,
Cambridge University Press, Cambridge, 1994.

\end{thebibliography}

\end{document}
